%%%%%%%%%%%%%%%%%%%%%%%%%%%%%%%%%%%%%%%%%%%%%%%%%%%%%%%%%%%%%%%%%%%%
%%%%%%%%%%%%%%%%%%%%%%%%%%%%%%%%%%%%%%%%%%%%%%%%%%%%%%%%%%%%%%%%%%%%

\chapter{Acknowledgments}


A 15-year history summary.

\textbf{Episode I}. The \webobs project was born in September 2000 when \textit{François Beauducel} has been assigned to the Guadeloupe volcanological observatory in Lesser Antilles. First ideas of an integrated monitoring and management system have risen thanks to fruitful discussions with \textit{Christian Anténor-Habazac}, \textit{Jean-Christophe Komorowski} and \textit{Stéphane Acounis}. Quickly (and dirtily) developed in about a year of sparse hours, a first version of \webobs was presented in Paris on January 2002 \citep{beauducel2002qes}, containing most of the present content: station files, networks, automatic graphs for seismic, deformation, geochimia and weather stations, shared calendar... During the first two years of the project, there was a single developer, moreover, a scientist during its overtime work and not a dedicated computer specialist!

\textbf{Episode II}. From August 2002 to December 2005, \textit{Didier Mallarino} was the first computer engineer who invested a part of its time to improve the codes and configuration files, especially by developing more robust and flexible Perl GCI scripts \citep{beauducel2004webovs,beauducel2005wim,beauducel2006sov}. In 2004, a version of \webobs has been partially duplicated and adapted for a public website (CDSA).

\textbf{Episode III}. From 2006 to 2010, some code improvements where made by a second computer engineer \textit{Alexis Bosson}: particularly, the system was internationalized and a first effort was made to integrate \webobs with observatory acquisition chain seismology standards \citep{beauducel2010webobs}. During these years, the system worked in a relatively stable production state, and it was adapted and partly installed in different observatories: Paris (thanks to \textit{François Truong} \citep{truong2009magis}), Addis-Abeba (thanks to \textit{Alexandre Nercessian}), Martinique (thanks to \textit{Jean-Marie Saurel} and \textit{Benoît Costes}), Montserrat (thanks to \textit{Alexis Bosson} and \textit{Roderick Stewart}) and later in 2012 at La Réunion (thanks to \textit{Patrice Boissier} and \textit{Philippe Kowalski}).

\textbf{Episode IV} (\textit{A New Hope}). In 2012, \webobs obtained its first dedicated funding support from the French Ministry of Ecology, thanks to \textit{Steve Tait}, \textit{Arnaud Lemarchand} and \textit{Pierre Agrinier}. A very significant contribution has been made by \textit{Didier Lafon}, the first computer engineer working 100\% on the project. Taking advantage of 10 years of production feedback, he reassessed the whole coding concept, improved and standardized the codes, made library modules and administration tools, wrote technical documentation, put all this under a versioning control system and built the first Linux installation package. This allowed to install a first alpha and beta version at Merapi observatory (thanks to \textit{Ali A. Fahmi}), then the same codes in Guadeloupe, Martinique and La Réunion observatories, and start a real collaborative development. During this last period, we welcomed additional contributors, as developers or end-users: \textit{Xavier Béguin}, \textit{Jean-Marie Saurel}, \textit{Stephen Roselia}, \textit{Patrice Boissier}, \textit{Laura Henriette}, and of course all the observatory teams under the direction of \textit{Jean-Bernard de Chabalier}, \textit{Valérie Clouard}, \textit{Andrea Di Muro}, \textit{Nicolas Villeneuve}, \textit{Céline Dessert}, \textit{Aline Peltier}, \textit{Roberto Moretti}.

\textbf{Episode V}. In October 2018, \webobs code has been released on \url{https://github.com/IPGP/webobs}.